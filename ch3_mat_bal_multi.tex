\documentclass[journal=jpcbfk,manuscript=article]{achemso}
%\usepackage{geometry}
% \geometry{a4paper,total={170mm,257mm},left=20mm,top=10mm,}
\usepackage{graphicx}
\rmfamily
\mdseries
\usepackage{amssymb,amsmath}
\usepackage{catchfilebetweentags}
\usepackage[utf8]{inputenc}
\usepackage[version=3]{mhchem}
\usepackage[T1]{fontenc}
\newcommand*\mycommand[1]{\texttt{\emph{#1}}}
\author{Kaustubh Rane}
\email{kaustubhrane@iitgn.ac.in}
\phone{0091 8433596760}
\affiliation{Chemical Engineering, Indian Institute of Technology, Gandhinagar, Palaj, Gujarat, India - 382355}
\title{Ch. 2 Material Balance on Multiple Processes}
\begin{document}

\maketitle

\section{3.1 Terminology}

When there are multiple processes, there are streams that go from one process to another (intermediate streams), and there are streams that come into a process or go out from the process. We use the following terminology for explanation:

\paragraph \noindent 1. Complete process: Including the intermediate streams

\paragraph \noindent 2. Combined process: We only consider the strems that enter from outside and leave to the outside. We do not consider the intermediate streams.

\section{3.2 General strategy}

\paragraph \noindent 1. Read the problem statement completely!

\paragraph \noindent 2. Draw the flow-chart

\paragraph \noindent 3. Identify the streams of interest.

\paragraph \noindent 4. Identify the total number of independent material balances

\paragraph \noindent 5. Tabulate the known data and identify the unknown variables.

\paragraph \noindent 6.  Determine the degree of freedom ($df$) for the complete process to know whether all the unknowns can be evaluated.

\paragraph \noindent 7. Determine the degree of freedom ($df$) for the combined process

\paragraph \noindent 8. If $df_{combined} = 0$, solve the equations to get the unknowns 

\paragraph \noindent 9. Calculate $df$ for the indivudual processes of the appropriate sub-groups of the complete process. Consider the variables determined in the previous step as "known" while performing the $df$-analysis.

\paragraph \noindent 8. Solve the equations for the processes or their groups that have $df=0$

\section{3.3 Important points}

\paragraph \noindent 1. Be careful while calculating the $df$ for the processes where a stream is split in to two such that each "daughter" stream has the same composition as the feed.

\paragraph \noindent 2. If a component is present in only some streams of a process, and if those streams occur in another process too, the material balance for the component should be only considered once while calculationg the $df$ of  the complete process.This situation is mainly encountered when a component (solvent or catalyst) is recycled without adding the fresh component.

\section{Student responses}

\subsection {14110151	Navdeep Prakash}

\subsection {15110034	Avinash Joy Bara}

\subsection {15110047	Deepti Gautam}

\subsection {15110133	Suresh Kumar}

\subsection {15110145	Vijendra Maurya}

\subsection {16110001	Abhavya Chandra}

\subsection {16110003	Abhishek Dubey}

\subsection {16110033	Bhumika Sandilya}

\subsection {16110036	Buditi Prudhvi}

\subsection {16110056	Gameti Nirav}

\subsection {16110071	Kamle Mayank Shrikant}

\subsection {16110077	Khili Khamesra}

\subsection {16110086	Lakhan Agrawal}

\subsection {16110088	Manjot Singh}

\subsection {16110109	Patel Milanbhai}

\subsection {16110127	Rahul Shakya}

\subsection {16110131	Raman}

\subsection {16110139	Ritik Jain}

\subsection {16110140	Rohan Gupta}

\subsection {16110153	Shubham Sankhla}

\subsection {16110156	Singh Shivam}

\subsection {16110158	Sourabh Saini}

\subsection {16110159	Spand Bharat Mehta}

\subsection {16110160	Sparsh Jain}

\subsection {16110173	Varsha Singh}

\subsection {16110179	Yash Makwana}


%\bibliographystyle{aipauth4-1}
%\bibliography{ch12references}


\end{document} 